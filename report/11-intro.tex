\chapter*{\hfill{}ВВЕДЕНИЕ\hfill{}}%
\label{cha:vvedenie}
\addcontentsline{toc}{chapter}{ВВЕДЕНИЕ}

Проблема обеспечения безопасности в сфере информационных технологий возникла в тот же момент, когда появились сами информационные технологии.

Корпорации, которые связаны с кибербезопасностью, тратят огромные суммы денег на разработку новых методов обнаружения и предотвращения атак на информационные системы. Вместе с этим существует большое число людей, которые намерено занимаются взломом компьютерных систем и разработкой вредоносного программного обеспечения для достижения совершенно различных целей. Одним из подходов в разработке вредоносного по являются руткиты.

Чаще всего основной целью руткитов является сокрытие вредоносного программного обеспечения, модификация и сокрытие данных, подмена системных вызовов, кража пользовательской информации. Но помимо вредоносных руткитов, также довольно часто можно встретить и те, назначение которых~--- предоставлять пользователю полезную функциональность, например, блокировка устройства или удаление конфиденциальных данных в случае кражи оборудования. Также стоит упомянуть о том, что большое множество различного антивирусного программного обеспечения реализовано схожим образом, что и руткиты. Целью таких руткитов является обнаружение других вредоносных руткитов или любого другого вредоносного программного обеспечения.

Целью данной работы является реализация руткита для сокрытия процессов и сетевых сокетов.

Для достижения поставленной цели необходимо решить следующие задачи:
\begin{itemize}
    \item
        изучение подходов к реализации руткитов;
    \item
        изучение исходного текста ядра;
    \item
        определение функциональности реализуемого руткита;
    \item
        исследование механизмов отображение процессов и сетевых сокетов;
    \item
        реализация руткита.
\end{itemize}
